\documentclass[../main.tex]{subfiles}

\begin{document}

\section{Richiami di statistica}
\begin{definition}[Partizione di un insieme]
    Una partizione dell'insieme $H$ è una famiglia di sottoinsiemi $\{ H_{1},H_{2},\cdots H_{k}\}$ che soddisfa le seguenti proprietà:
    \begin{enumerate}
        \item $H_{i}\cap H_{j}=\emptyset$ per ogni $i\neq j$;
        \item $\bigcup_{i=1}^{k}H_{i}=H$.
    \end{enumerate}
\end{definition}
In altre parole abbiamo detto che:
\begin{itemize}
    \item La famiglia $\{ H_{1},H_{2},\cdots H_{k}\}$ è detta \textbf{disgiunta} se $H_{i}\cap H_{j}=\emptyset$ per ogni $i\neq j$.
    \item La famiglia $\{ H_{1},H_{2},\cdots H_{k}\}$ è detta \textbf{completa} se $\bigcup_{i=1}^{k}H_{i}=H$.
    \item La famiglia $\{ H_{1},H_{2},\cdots H_{k}\}$ è detta \textbf{partizione di H} se è disgiunta e completa. 
\end{itemize}
Sia $\{ H_{1},H_{2},\cdots H_{k}\}$ un partizione di $H$,$P(H)=1$ e sia $E$ un evento specifico. Allora gli assiomi di probabilità ci dicono che:
\begin{itemize}
    \item \begin{equation}\label{eq:prob tot}
        \sum_{i=1}^{k}P(H_{i})=1.
    \end{equation}
    \item \begin{equation}\label{eq:prob marg}
        P(E)=\sum_{i=1}^{k}P(E\cap H_{i})=\sum_{i=1}^{k}P(E|H_{i})P(H_{i}).
    \end{equation}
\end{itemize}
Dove \ref{eq:prob tot} è detta \textbf{Regola di probabilità totale} e \ref{eq:prob marg} è detta \textbf{Regola di probabilità marginale}.
\begin{definition}[Formula di Bayes]
\begin{equation}
    P(H_{j}|E)=\frac{P(E|H_{j})P(H_{j})}{P(E)}\stackrel{\ref{eq:prob marg}}{=}\frac{P(E|H_{j})P(H_{j})}{\sum_{i=1}^{k}P(E|H_{i})P(H_{i})}.
\end{equation}
\end{definition}
\end{document}